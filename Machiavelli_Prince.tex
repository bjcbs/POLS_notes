\documentclass[10pt]{article}

%geometry, formatting
\usepackage{geometry}[margins=1in]
\setlength{\parindent}{0pt}
\setlength{\parskip}{11pt}
%other packages
\usepackage{color}
\usepackage[dvipsnames]{xcolor}

%commands
\newcommand{\aside}[1]{{\color{Goldenrod}#1}}
\newcommand{\question}[1]{{\color{BurntOrange}#1}}
\newcommand{\comment}[1]{{\color{Cerulean}#1}}
\newcommand{\keyquote}[1]{{\color{BrickRed}#1}}

%%%% %%%% %%%% %%%% %%%% %%%% %%%% %%%% %%%% %%%% %%%% %%%% %%%% %%%% %%%% %%%%

\begin{document}
\hfill Brent Jacobs

Notes on \textsl{The Prince}, Niccol\`{o} Machiavelli, translated by
Harvery C. Mansfield (2nd Edition)

%%%% Introduction
\textbf{Introduction by Harvey C. Mansfield}
\begin{itemize}
\item 
    ``on the basis of Machiavelli's saying in Chapter 15 that we should
    take our bearings from `what is done' rather than from `what should be 
    done,' they [scholars of Machiavelli] conclude that he was a forerunner of
    modern political science, which is not an evil thing because it
    merely tells us what happens without passing judgment.'' (p. viii)

\item 
    ``What is at issue in the question of whether Machiavelli was
    `Machiaveliian'''? (p. ix)

\item 
    ``he never mentions---not in \textsl{The Prince}, or in any of his
    works---natural justice or natureal law'' (p. xii)

\item 
    ``Since nature or God does not support human justice, men are in need
    of a remedy; and the remedy is the prince, especially the new prince.''
    (p. xii)

\item 
    New prince should prefer the common people over the aristocracy,
    since they are easier to please, harder to eliminate, and less
    likely to oppose him (p. xix, referencing Ch. 7)
11
\item 
    \textsl{The Prince} was written ~1513, dedicated to Lorenzo de' Medici.
    Different versions exist, so no ``original'' exists.
\end{itemize}

%%%% Chapter 1
\textbf{I. How Many Are the Kinds of Principalities and in What Modes Are 
They Acquired?}
\begin{itemize}
\item
    ``All states [...] have been either republics or principalities. 
    The principalities are either hereditary [...] or they are new. 
    The new ones are either altogether new [...] or they are like
    members added to the hereditary state of the prince who acquires
    them'' (page 5)
\item ``Dominions so acquires are either accustomed to living under a prince
    or used to being free'' (page 6)
\end{itemize}

%%%% Chapter 2
\textbf{II. Of Hereditary Principalities} 
\begin{itemize}
\item
    NM won't ``reason[] on republics'' since he does so in
    \textsl{Discourses on Livy}
\item 
    It is easy to maintain an inherited principality, barring incompetence 
    or extraordinary circumstances (p. 6-7)
\item 
    Hereditary rulers have to ruffle fewer feathers, so it is easier to
    be viewed favorably (p. 7)
\end{itemize}

%%%% Chapter 3
\textbf{III. Of Mixed Principalities}
\begin{itemize}
\item
    ``But the difficulties reside in the new principality.'' (p. 7)
\item
    Mixed: not wholly new, but added onto an existing 
    principality/state/dominion
\item
    In acquiring a new territory, Prince inevitably offends the residents,
    and ``cannot keep as friends those who have put you there because
    you cannot satisfy them in the mode they had presumed'' (p. 8).
\item
    If a neighboring province has a similar culture to a well-established
    country, it is easier to integrate, ``especially if they are not used
    to living free.'' (p. 9) NM gives regions of `France' as an example.
    \comment{Why fear Russia annexing Ukraine? It could be successful!}
    Other tips: kill all heirs, and don't alter the laws or the taxes,
    to allow for a smooth transition.
\item
    To acquire and hold culturally disparate provinces,
     ``one needs to have great fortune and great industry'' (p. 9-10).
    Tip: Go live there in person! Like the Turks in Greece! 
    Problems can be nipped in the bud. The people can love the readily
    accessible prince... or at least for him more. Aggressors are less
    likely to invade.
\item   
    ``The other, better remedy is to send colonies''. Just a couple small
    ones though. Since they're small, not too many
    people need to be ``offended'' by the confiscation of their land. 
    Keep those displaced poor, and disperesed. 
    The rest will be thankful they were spared, but fearful the same could
    happend to them. 
    (p. 10). 
\item 
    The alternative to colonies is troops, but this is expensive
    and offends many as ``one's army moves around for lodgings.''
    ``Everyone feels this hardship, and each becomes one's enemy,
    and these are enemies that can harm one since they remain, 
    though defeated, in their homes.'' (p. 10-11).
\item 
    ``men should either be caressed or eliminated, because they avenge
    themselves for slight offenses but cannot do so for grave ones; so
    the offense one does to a man should be such that one does not fear
    revenge for it.'' (p. 10-11)
\item 
    If ruling a ``disparate'' land, keep the neighboring states weak,
    subservient, and/or dependent. Do not allow a different powerful
    foreignor to take hold next door. The inhabitents of the weak 
    provinces will welcome the powerful foreignor, since they loathe
    whomever has power over them. (p. 11) 
\item
    \keyquote{
    ``Thus, the Romans, seeing incoveniences from afar, always found
    remedies for them and never allowed them to continue so as to 
    escape a war, because they knew that war may not be avoided but 
    is deffered to the advantage of other.''} (p. 12-13)
\item
    \comment{NM relies on reason \textsl{and} evidence. He cites 
    the Greeks, the Romans, and recent French kings' dealings with
    Italian states. Are his examples the best? I don't know enough to 
    judge. More importantly, there is the assumption that historical
    evidence must exist to support his theories. Very different
    from Plato's \textsl{The Republic}, which theorized about
    a state that had never existed. Looking to the past for principles
    that explain how states and governments function---I can see know
    why NM is considered ``modern'' and ``the first political 
    scientist.''}
\item
    ``whoever is the cause of someone's becoming powerful is ruinied;
    for that power has been caused by him either with industry or with
    force, and both the one and the other of these two are suspect to 
    whoever has become powerful.'' \question{not sure?} (p. 16)
\end{itemize}

%%%% Chapter 4
\textbf{IV. Why the Kingdom of Darius Which Alexander Seized Did Not Rebel
from His Successors after Alexander's Death}
\begin{itemize}
\item
    In some principalities, the prince is above other hereditary lords
    (e.g. France) who enjoy privileges and the ``love'' of their
    people, in others (e.g. `the Turk') a single monarch governs 
    through ministers and governors who serve at the monarch's pleasure.
    The former is easier to conquer but harder to hold. And vice-versa.  
\end{itemize}

%%%% Chapter 5
\textbf{V. How Cities or Principalities Which Lived by Their Own Laws 
before They Were Occupied Should By Administered}
\begin{itemize}
\item
    ``When those states that are acquired [...] are accustomed to
    living by their own laws and in liberty, there are three
    modes for those who want to hold them: first, run them;
    second, go there to live personally; third, let them live
    by their laws, taking tribute from them and creating
    within them an oligarchical state which keeps them friendly
    to you.''
\item
    States formerly ruled by another prince cannot coherently rebel 
    and become a free state; whereas former free states will remember
    they were once free and could at any point rebel and reorganize
    as a free state.
\end{itemize}

%%%% Chapter 6
\textbf{VI. Of New Principalities That Are Acquired through One's
Own Arms and Virtue}
\begin{itemize}
\item
    Since once can never imitate fully, one should imitate the greatest
    rulers, leaders, etc. so that, when one fails to match their full
    greatness, he is still not so far off (p. 21-22). 
\item 
    ``I say, then, that in altogether new pricipalities, where there is a new
    prince, one encounters more or less difficulty in maintaining them
    according to whether the one who acquires them is more or less
    virtuous.'' (p. 22)
\item
    Best examples: ``Moses, Cyrus, Romulus, Theseus, and the like.'' But
    we can ignore reasoning about Moses, because God was pulling the strings
\item
    Setting up a new state is tricky, since all who benefited by the old
    system are strong enemies and those who might benefit from the 
    new system are ``lukewarm defenders''. (p 23)
\item
    Being armed is very important. (p. 23)
\end{itemize}

%%%% Chapter 7
\textbf{VII. Of New Principalities That Are Acquired by Others'
Arms and Fortune}
\begin{itemize}
\item
    Those who becomes princes with little difficulty, since they relied on
    fortune and not their own virture, have a difficult time maintaining
    their positions. (p. 25)
\item
    New systems and institutions lack roots. (p. 26)
\item
    \keyquote{``And whoever believes that among great personages new
    benefits will make old injuries be forgotten deceives himself.''}
    (p. 33) 
\end{itemize}

%%%% Chapter 8
\textbf{VIII. Of Those Who Have Attained a Principality though Crimes}
\begin{itemize}
    \item 
        ``cruelties badly used or well used'' (p. 37)
    \item
        Well-used cruelties are brief, and serve a particular purpose
        (p. 37-38)
\end{itemize}

%%%% Chapter 9
\textbf{IX. Of the Civil Principality}
\begin{itemize}
    \item
        ``when a private citizen becomes prince of his fatherland, not
        through crime or other intolerable violence but with the 
        support of his fellow citizens \ldots''    
    \item 
        ``one ascends to this principality either with the 
        support of the people or with the support of the great.''
        The people will put a new leader in, to avoid being oppressed
        by the great, while the great will put a new leader in, hoping
        to oppress the people. (p. 39) 

        Depending on which motive prevails, three possible outcomes:
        \begin{itemize}
            \item \textsl{Principality.} 
                Put in place by the great when they struggle to control 
                the masses, or by the masses when they struggle to resist
                the great.
            
                Difficult to maintain one's position after being placed by 
                the great. First, many
                similarly-powerful may challenge him. Plus, the masses
                simply don't want to be oppressed, whereas it is much
                harder to please the great. Further, it is easier to
                guard against the machinations of a few than to calm
                the anger of the many.    

            \item \textsl{Liberty.}
                ?
            \item \textsl{License.}
                ?
        \end{itemize}
    \item
        If installing magistrates, then at the mercy of them. (p. 42)
    \item
        \keyquote{ 
        ``so a wise prince must think of a way by which his citizens, always
        and in every quality of time, have need of the state and of himself;
        and then they will always be faithful to him.''} (p. 42)
\end{itemize}

%%%% Chapter 10
\textbf{X. In What Mode the Forces of All Principalities Should 
Be Measured}
    Two cases: princes whose states are large enough to fights battles,
    ad princes who can merely defend their towns. The latter should 
    fortify their cities well (e.g. cities of Germany).

%%%% Chapter 11
\textbf{XI. Of Ecclesiastical Principalities}
\begin{itemize}
\item 
    Are maintained easily, ``for they are sustained by orders
    that have grown old with religion,'' so princes needn't
    actively govern and subjects do not rebel (p. 45)   
\end{itemize}

%%%% Chapter 12
\textbf{XII. How Many Kinds of Miliary There Are and Concerning
Mercenary Soldiers}
\begin{itemize}
\item 
    \keyquote{
    ``The principal foundations that all states have [...] are good laws
        and good arms. And because there cannot be good laws where there are
        not good arms, and where there are good arms there must be good laws},
    I shall leave out the reasoning on laws and shall speak of arms.'' 
    (p. 48) 
\item 
    Arms ``are either his own or marcenary or auxillary or mixed.'' 
    (p. 48)
\item
    ``Mercenary and auxillary arms are useless and dangerous; and if one 
    keeps his state founded on mercenary arms, one will never be firm or 
    secure'' (p. 48)
\end{itemize}

%%%% Chapter 13
\textbf{XIII. Of Auxiliary, Mixed, and One's Own Soldiers}
\begin{itemize}
\item 
    Auxiliary arms belong to another prince or power
\item 
    ``These are can be useful and good for themselves, but for whoever
    calls them in, they are almost always harmful, because when they 
    lose you are undone; when the win, you are left their prisoner.''
    (p. 54) \aside{South Korea!}
\item
    ``A wise prince [...] has preferred to lsoe with his own than to    
    win with others, since he judges it no true victory that is
    acquired with alien arms.'' (p. 55)
\item
     examples, examples, \ldots
\end{itemize}

%%%% Chapter 14
\textbf{XIV. What a Prince Should Do Regarding the Military}
\begin{itemize}
\item
    \keyquote{
    ``Thus, a prince should have no other object, nor any other thought, not 
    take anything else as his art but that of war and its orders and
    discipline; for that is the only art which is of concern to
    one who command.''} (p. 58)
\item
    In peacetime, a prince should keep the armies in order and go out
    hunting to toughen up and to learn the landscape. Further,
    study the exploits of great military-men. (p. 59-60)
\end{itemize}

%%%% Chapter 15
\textbf{XV. Of Those Things for Which Men and Especially Princes
Are Praised or Blamed}
\begin{itemize}
\item
    \keyquote{
    ``But since my intent is to write something useful to whoever 
    understands it, it has appeared to me more fitting to go
    directly to the effectual truth of the thing than to the 
    imagination of it.''} (p. 61)
\item    
    To be successful, one cannot simply imagine an ideal state, 
    but must study what has actually happened. (p. 61)
\item
    It is impossible for a prince to be a wholly good person,
    so at least he can ``be so prudent as to know how to avoid
    the infamy of those vices that would take his state from him.'' 
    But, ``one should not care about incurring
    the fame [note: infamy?] of those vices without which
    it is difficult to save one's state.'' (p. 62)
\item 
    Essentially: A prince should strive to be perceived as a good person, 
    so long as it helps him hold power. 
\end{itemize}

%%%% Chapter 16
\textbf{XVI. Of Liberality and Parsimony}
\begin{itemize}
\item   
    ``it would be good to be held liberal; nonetheless, liberality,
    when used so that you may be held liberal, hurts you. For if it 
    is used virtuously and as it should be used, it may not be 
    recognized'' (p.62-63)
\item 
    Being openly liberal would drain a prince's resources, forcing
    him to burden the people with taxes, preventing any reputation
    for liberality. (p. 63)
\item
    Instead, the prince should be parsimonious, not caring about a 
    reputation for meanness. Over time, those around him will 
    appreciate that the prince can maintain himself and the state
    without burdening the people. (p. 63)
\item
    ``a prince should esteem it little tp incur a name for meanness,
    because this is one of those vices which enable him to rule.''
    (p. 64) 
    A repuation for meanness is ``infamy without hatred.''
    (p. 65)
\end{itemize}

%%%% Chapter 17
\textbf{XVII. Of Cruely and Mercy, and Whether It Is Better to Be Loved
Than Feared, or the Contrary}
\begin{itemize}
    \item 
    \keyquote{
    ``each prince should desite to be held merciful and not 
    cruel; nonetheless he should take care not to use this mercy
    badly.''} (pg. 65)
\item 
    Disorder harms many, whereas executions harm but individuals. (p. 66)
\item 
    New states are disordered, demanding harsh measures. (p. 66)
\item 
    \keyquote{
    ``one would want to be both [loved] and [feared]; but because it is
    difficult to put them together, it is much safer to be feared than 
    loved''} (p. 66)
\item 
    \keyquote{
    ``friendships that are acquired at a price and not with greatness
    and nobility of spirit are bought, but they are not owned and when
    the time comes they cannot be spent.'' 
    } (p. 66)
\item
    ``love is held by a chain of obligation, which, because mean are
    wicked, is broken at every opportunity for their own utility, but
    fear is held by a dread of punishment that never forsakes you.''
    (p. 67)
\item 
    \keyquote{
    ``being feared and not being hated can go together very well.''
    } Thus, a prince should abstain from the property (and
    women!) of others, and not take life without good reason. But
    especially not property: ``men forget the death of a father
    more quickly than the loss of a patrimony.'' (p. 67)
\item
    When leading an army, one must be cruel, terrifying, and in
    other respects, virtuous. (p. 67)
\end{itemize}

%%%% Chapter 18
\textbf{XVIII. In What Mode Faith Should Be Kept by Princes}
\begin{itemize}
\item
    \keyquote{
    ``it is necessary for a prince to know well how to use the 
    beast and the man.''
    } (p. 69)
\item
    It is useful to \textsl{appear} to be religious and pious,
    but harmful to strictly be so. (p. 70)
\item
    The many will simply see how you present yourself---ideally,
    pious and good-natured---and the few that know the true 
    nature of you---deceiving, cunning, evil---will not speak
    against the opinion of the many. (p. 71)
\item
    ``So let a prince win and maintain his state: the means will
    always be judged honorable, and will be praised by everyone.''
    (p. 71)
\end{itemize}

%%%% Chapter 19
\textbf{XIX. Of Avoiding Contempt and Hatred}
\begin{itemize}
\item 
    ``What makes [a prince] contemptible is to be held variable,
    light, effeminate, pusillanimous, irresolute, from which a prince
    should guard himself'' (p. 72)
\item
    Two types of threats to guard against:
    \begin{itemize}
    \item 
        \textsl{External.} One is defended from external threats if
        ``one id defended with good arms and good friends; and if one
        has good arms, one will always have good friends.'' (p. 72)
    \item
        \textsl{Internal.} ``From this the prince may secure himself
        sufficiently if he avoids being hated or despised [...]
        For whoever conspires always believes he will satisfy the people
        with the death of the prince, but when he believes he will offend
        them [...] the difficulties on the side of the conspirators are
        infinite.'' (p. 73) 
        \comment{I think this underestimates how much a small group
        of malcontents can mininterpret the will of the whole.}
    \item 
        Goes through a whole bunch of Roman emporers that could be
        proposed as counter examples, and explains why they aren't.
        Interesting rhetorically, compared to the ancients, but I've
        deferred considering the details.
    \end{itemize}
\end{itemize}

%%%% Chapter 20
\textbf{XX. Whether Fortresses and Many Other Things Which Are Made and 
Done by Princes Every Day Are Useful or Useless}
\begin{itemize}
\item
    Princes have, historically, done many things in opposite ways
    from each other. Each case is particular, but NM will speak to
    the general principles. (p. 83)
\item 
    When forming a new state, arming subjects makes them loyal. 
    Disarming subjects offends them,
    generated hatred. Worse: unarmed subjects demands a mercenary 
    force, always bad. (p. 83)
\item
    When a new state is added to a prince's existing one, then the new
    state should be disarmed. This keeps the armed within one's own
    state, firmly under one's control. (p. 83) 
\item
    Internal divisions should be avoided. While divisions keep subjects
    more easily managed in peacetime, they can disasterous once war 
    comes. (p. 83-84)
\item
    A wise prince encourages some enmity against and obstacles for him,
    so that he has something to overcome, then achieving more greatness.
    (p. 85)
\item
    Those who initially oppose a new prince can likely become the most
    useful and reliable, once won over. (p. 85)
\item
    NM seems ambivalent on the usefullness of fortresses, thinking it
    depends on the situation too much? Says the hatred of the people
    matters more (p. 86-87) ?
\end{itemize}

%%%% Chapter 21
\textbf{XXI. What a Prince Should Do to Be Held in Esteem}
\begin{itemize}
    \item 
        Should declare as friend or enemy, not remain neutral (p. 80)
    \item 
        Shouldn't go into battle with a greater power, except
        sometimes? (p, 88-89)
    \item 
        patronize the arts, entertain citizens with festivals \ldots \ldots
\end{itemize}

%%%% Chapter 22
\textbf{XXII. Of Those Whom Princes Have as Secretaries}
The choice of ministers reflects heavily on a prince (p. 92).
Good ministers are loyal, thinking only of and sharing burdens
with the prince (p. 93).

%%%% Chapter 23
\textbf{XXIII. In What Mode Flatterers Are to Be Avoided}
\begin{itemize}
\item
    It is difficult to balance avoiding flattery with not being
    contemptible. If mean feel they can speak the truth to the prince
    without offending, flatterers will be avoiding; but, everyone
    speaking the truth shows a lack of reverence. Instead, ``a 
    prudent prince must [...] choos[e] wise men in his state; and
    only to these should he give freedom to speak the truth to him''
    (p. 94). 
\item
    Rather than accept unsolicited advice, the prince should ask for 
    counsel. But, a prince must make his own decisions and stick
    to them (p. 94). 
\item
    \keyquote{
    ``good counsel, from wherever it comes, must arise from
    the prudence of the prince, and not the prudence of the
    prince from good counsel.''} (p. 95)
\end{itemize}

%%%% Chapter 24
\textbf{XXIV. Why the Princes of Italy Have Lost Their States}
Only two pages, this chapter basically summarizes that the various
reasons Italian princes have lost their states can be indentified
in the couple dozen precedning chapters. 

%%%% Chapter 25
\textbf{XXV. How Much Fortune Can Do in Human Affairs, and in What
Mode It May Be Opposed}
\begin{itemize}
\item
    Fortune plays a large role in human affairs, but not so large one
    should discount the effects of human efforts. Fortune harms
    most where virtue is absent. Virtue guards against the whims of
    fortune.
\item
    Princes who rise through fortune, but do not rely on virtue to
    maintain their position, promptly fall (p. 99).
\item
    To be successful, one must be flexible in their ways, 
    their nature (p. 100-101).
\end{itemize}

%%%% Chapter 26
\textbf{XXVI. Exhortations to Seize Italy and to Free Her from the
Barbarians}
\begin{itemize}
\item
    There has never been a better time for a new prince in Italy!
    (p. 101-102)
\item 
    In fact, you, to whom this book is dedicated, must! (p. 103)
\end{itemize}

%%%% Overall notes
\textbf{Overall notes}
\begin{itemize}
\item
    Omitted from my notes are numerous historical examples. I am not 
    familiar with most of them, so cannot readily judge their quality.
    More significant is his reliance on examples, rather than 
    ideals. See Ch. 15, p. 61.
\item
    NM even predicts proposed counterexamples, and attempts to counter
    them. Example on page 75: `` Since I want, therefore, to respond to these
    objections.''
\end{itemize}
\end{document}
