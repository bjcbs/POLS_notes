\documentclass[11pt]{article}

%geometry, formatting
\usepackage[margin=1.5in]{geometry}
\setlength{\parindent}{0ex}
\setlength{\parskip}{11pt}
%other packages
\usepackage{color}
\usepackage[dvipsnames]{xcolor}
\usepackage{enumitem}

%commands
\newcommand{\aside}[1]{{\color{Goldenrod}#1}}
\newcommand{\question}[1]{{\color{BurntOrange}#1}}
\newcommand{\comment}[1]{{\color{Cerulean}#1}}
\newcommand{\keyquote}[1]{{\color{BrickRed}#1}}

\newcommand{\sectiontitle}[1]{{\Large {\bfseries #1}}}
\newcommand{\chapter}[1]{{\bfseries #1}}

\newcommand{\p}{page }
%%%% %%%% %%%% %%%% %%%% %%%% %%%% %%%% %%%% %%%% %%%% %%%% %%%% %%%% %%%% %%%%

%“Epistle Dedicatory”; “Introduction”; Book I, Chapters 1-7, 10-21, 24, 26-32; 43, 45-47; “A Review and Conclusion”

\begin{document}
\hfill Brent Jacobs

{\Large Notes on Thomas Hobbes' \textsl{Leviathan}} 

\sectiontitle{Epistle Dedicatory}
\begin{itemize}
\item
    ``with those that contend on one side for too great Liberty,
    and on the other side for too much Authority, 'tis hard
    to passe between the points of both unwounded.'' (\p 75)
\item
    Hobbes knows his work will be controversial, attacked on
    multiple sides.
\end{itemize}

\sectiontitle{The First Part: Of Man}

\chapter{1. Of Sense}
\begin{itemize}
\item
\end{itemize}

\chapter{2. Of Imagination}

\chapter{3. Of the Consequence or Train of Imaginations}

\chapter{4. Of Speech}

\chapter{5. Of Reason and Science}

\chapter{6. Of the interiour Beginnings of Voluntary Motions, commonly
called the Passions; And the Speeches by which they are expressed}

\chapter{7. Of the End or Resolutions of Discourse}

\chapter{10. Of Power, Worth, Dignity, Honour, and Worthinesse}

\chapter{11. Of the Difference of Manners}

\chapter{12. Of Religion}

\chapter{13. Of the Naturall Condition of Mankind as concerning their Felicity
and Misery}

\chapter{14. Of the first and second Naturall Lawes, and of Contract}

\chapter{15. Of other Lawes of Nature}

\chapter{16. Of Persons, Authors, and things personated}

\sectiontitle{The Second Part: Of Common-Wealth}

\chapter{17. Of the Causes, Generation, and Definition of a Common-wealth}

\chapter{18. Of the Rights of Soveraignes by Institution}

\chapter{19. Of severall Kinds of Common-wealth by Institution;
and of Succesion to the Soveraign Power}

\chapter{20. Of Dominion Paternall, and Despoticall}

\chapter{21. Of the Libery of the Subjects}

\chapter{24. Of the Nutrition, and Procreation of a Common-wealth}

\chapter{26. Of Civill Lawes}

\chapter{27. Of Crimes, Excuses, and Extenuations}

\chapter{28. Of Punishmnts, and Rewards}

\chapter{29. Of those things that Weaken, or tend to the Dissolution of a 
Common-wealth}

\chapter{30. Of the Office of the Soveraign Representative}

\chapter{31. Of the Kingdome of God by Nature}

\sectiontitle{The Third Part: Of a Christian Common-wealth} 

\chapter{32. Of the PRinciples of Christian Politiques}

\chapter{43. Of what is Necessary for mans Reception into the Kingdome
of Heaven}

\sectiontitle{The Fourth Part: Of the Kinddome of Darknesse} 

\chapter{45. Of D\ae monology, and other Reliques of the 
Religion of the Gentiles}

\chapter{46. Of Darknesse from Vain Philosophy, and Fabulous Traditions}

\chapter{47. Of the Benefit proceeding from such Darknesse; and
to whom it accreweth}

\sectiontitle{A Review and Conclusion}

\end{document}
