\documentclass[11pt]{article}

%geometry, formatting
\usepackage[margin=1.5in]{geometry}
\setlength{\parindent}{0ex}
\setlength{\parskip}{11pt}
%other packages
\usepackage{color}
\usepackage[dvipsnames]{xcolor}
\usepackage{enumitem}

%commands
\newcommand{\aside}[1]{{\color{Goldenrod}#1}}
\newcommand{\question}[1]{{\color{BurntOrange}#1}}
\newcommand{\comment}[1]{{\color{Cerulean}#1}}
\newcommand{\keyquote}[1]{{\color{BrickRed}#1}}

\newcommand{\book}[1]{{\Large {\bfseries Book #1}}}
\newcommand{\discourses}[1]{{\bfseries Discourses #1.}}
\newcommand{\discourse}[1]{{\bfseries Discourse #1.}}

%%%% %%%% %%%% %%%% %%%% %%%% %%%% %%%% %%%% %%%% %%%% %%%% %%%% %%%% %%%% %%%%

\begin{document}
\hfill Brent Jacobs
\begin{center}
Paragraph on Machiavelli's \textsl{Discourses}
\end{center}
What does Machiavelli see as essential to being a republic?
Specifically, how should power be distributed?

%In the beginning he establishes republics as distinct from Aristocracies and
%Democracies, but doesn't lay out criteria for the distinction.
Throughout the \textsl{Discourses}, he
considers republics with diverse constitutions and histories.
He is fond of the mixed government of Rome, but isn't opposed to the
quite different Venetian model either.

Who should control the levers of power, and whose opinions should matter?
At times, he either praises (Discourse 4) or laments
the effects of popular input.
In Discourse 53, he talks
about how the nobles of Rome had to keep the populace from ruining
themselves---and established this as a general principle.
In Discourse 34, Machiavelli says it is important to, in times of emergency,
transfer unquestionable power to a single executive, i.e. ``dictator.''
He also thinks it is best to entrust ``safeguarding liberty'' to an
``upper class'' to avoid ``endless quabbles and troubles in a republic''
(page 116).
While Machiavelli praises mechanisms through which the common people
can have influence, such a the Roman tribunals, he spends more space
talking about how leaders in a republic must control and calm the people.
%Public indictments are necessary to keep the people calm. Religion must
%be used to keep the people orderly.
% He doesn't see public opinion as a source of wisdom, but also considers
% it more reliable than the whims of a prince.
\end{document}
