\documentclass[11pt]{article}

%geometry, formatting
\usepackage{geometry}[margins=1in]
\setlength{\parindent}{0pt}

%other packages
\usepackage{color}
\usepackage[dvipsnames]{xcolor}

%commands
\newcommand{\aside}[1]{{\color{Goldenrod}#1}}
\newcommand{\question}[1]{{\color{BurntOrange}#1}}
\newcommand{\comment}[1]{{\color{Cerulean}#1}}


%%%% %%%% %%%% %%%% %%%% %%%% %%%% %%%% %%%% %%%% %%%% %%%% %%%% %%%% %%%% %%%%

\begin{document}

Notes on ``The Liberty of the Ancients Compared with that of the 
Moderns,'' Benjamin Constant (1816)

\begin{itemize}

\item Different senses of ``liberty'' were dear to ``ancient peoples''
than are ``precious to the modern nations''

\item  Constant wonders why, the form of representative goverment France 
now (1810s) enjoys---``the only one in the shelter of which we could
find some freedom and peace today''---``was totally unkown to
the free nations of antiquity.''

    - rejects other theorists' assertions that ancient governments 
    were representative

\item Ancient peoples count not ``feel the need for'' nor ``appreciate''
the advantages of modern representative government---becausae they
``desire[d] an entirely different freedom''

\item Ancients did not value \textsl{individual} freedoms. 
Instead, the emphasis was the freedom of the \textsl{collective} to,
through public deliberation, exercise sovereignty. But, this
``collective freedom'' demanded the ``complete subjection of the 
individual to the community.''
\aside{How does this relate to nations such as China. Do Chinese, in general,
have a different notion of freedom? Counterpoint: Taiwan}

\item Traces difference (partially) to: ancient states were small, and
had to constantly wage war to defend themselves, while modern Europe
is ``essentially homogeneous'' (?) and states 
have a ``uniform tendency [...] towards peace.''

\item Commerce supplanted war, as trade became a better way to aquire 
than violence. 
\begin{quote}
    ``War is all impulse, commerce, calculation. Hence it follows
    that an age must come in which commerce replaces war. We have 
    reached this age.''
\end{quote}
\comment{this... feels like too large of a jump to merit a `hence'}
``commerce does not, like war, leave in men's lives
intervals of inactivity.''

\item Eradiction of slavery in Europe. ``Without the slave 
population of Athens, 20,000 Athenians could never have
spent every day at the public square in discussions.''

\item As nations grow, the political importance of an individual citizen
diminishes.

\item Governments inherently handle business worse than individuals, 
so those involved in commerce must loathe goverment interference in
their private affairs

\item Spends a long time explaining how, while Athens is an exception,
it really is not?

\item Very interesting for whom he thinks ``liberty'', in either sense,
applies: free Republicans, and those who can reap the most reward from
commerce. 
\comment{Does the common man have liberty? Is he concerned with liberty?}

\item 
``No one has the right to tear the citizen from his country 
\comment{[Some European countries can]}, the owner away from
his possessions \comment{[Is this not counting eminent domain?]},
the merchant away from his trade \comment{[Licensing requirements!
Bar exams! Inspections!]}, the husband from his wife \comment{[Can the 
wife?]}, the father from his children \comment{[Ah, no CPS in 
nineteenth century France?]}, the writer from his studious mediations,
the old man from his accustomed way of life \comment{[okay?]}.'' [page 9]

\item Specific concern of author: Cencorship \`{a} la Rome. Apparently, fear
this was being implemented in France. Modern (to the author) Frenchmen 
were citing the Ancients to justify policies that, to a modern 
citizen, would be restrictions on liberty. 

\item ``Individual liberty, I repeat, is the true modern liberty'' (p. 11)

\item page 12: \question{Wait? Is he saying wealth under capitalism 
inevitably leads to legal impunity? Well, slight anachronism, but
still...} Okay, but at very least: only the wealthy have political
power when individual liberty is prioritized \question{Yes? This
is the implication?}

\item ``The danger of modern liberty is that, absorbed in the enjoyment
of our private independence, and in the puruit of our particular interests,
we should surrender our right to share in political power too easily.'' 
(p. 13) 

\item Through representatives, the modern man may still maintain 
political liberty. While individual liberty may lead to hapiness,
``political liberty is the most powerful, the most effective means
of self-development that heaven has given us.'' (p. 13) 


\end{itemize}

\end{document}
