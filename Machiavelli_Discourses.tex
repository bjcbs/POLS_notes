\documentclass[11pt]{article}

%geometry, formatting
\usepackage[margin=1.5in]{geometry}
\setlength{\parindent}{0ex}
\setlength{\parskip}{11pt}
%other packages
\usepackage{color}
\usepackage[dvipsnames]{xcolor}
\usepackage{enumitem}

%commands
\newcommand{\aside}[1]{{\color{Goldenrod}#1}}
\newcommand{\question}[1]{{\color{BurntOrange}#1}}
\newcommand{\comment}[1]{{\color{Cerulean}#1}}
\newcommand{\keyquote}[1]{{\color{BrickRed}#1}}

\newcommand{\book}[1]{{\Large {\bfseries Book #1}}}
\newcommand{\discourses}[1]{{\bfseries Discourses #1.}}
\newcommand{\discourse}[1]{{\bfseries Discourse #1.}}

\newcommand{\p}{page } 
\newcommand{\d}{disc. }
%%%% %%%% %%%% %%%% %%%% %%%% %%%% %%%% %%%% %%%% %%%% %%%% %%%% %%%% %%%% %%%%

\begin{document}
\hfill Brent Jacobs

Notes on Machiavelli, Discourses on Livy: 
Book I (entire);
Book II, discourses 1-14, 21-23, 28-29,
33; 
Book III, discourses 1-6, 35.
Editor: Bernard Crick.

{\Large{\bfseries Dedicatory letter.}}

I had thought, in NM's dedicatory letter for \textsl{The Prince}, he was
subtley mocking Lorenzo de Medici. Now I am convinced!

\book{I}

Deals with the development of Rome and its consititution.

{\bfseries Preface.}

While the nations of antiquity, especially Rome, are
greatly admired, few try to imitate it---or even believe they can. 
This is especially true in citizenship and statemenship. So, NM will
``write a commentary on all those book of Titus Livy'' to convice men
that similar greatness can be achieved today (a relative today). (\p 98-99) 

\discourses{1-10}

\begin{itemize}
\item
	``all cities are built either by natives of the place in which they are 
	built, or by people from elsewhere''(\p 100).
\item
	Free cities should be founded in fertile places, with many resources.
	But, since this might make men grow idle, strict laws must be enacted
	to ensure the men work diligently (\p 102-104). 
\item
	``I [...] shall speak only of those [cities] which have from the outset
	been far removed from any kind of external servitude, but, instead,
	have from the start been governed in accordance with their wishes, either
	as republics or principalities'' (\d 2, \p 104-105). 

	NM does not see being a principalities as incompatible with a city
	being ``governed in accordance with their wishes.''
\item
	States must be founded with good laws. Otherwise, it is unlikely a 
	good set of laws can ever later be instituted. (\d 2, \p 105)
\item 
	Common viewpoint: Principality $\rightarrow$ Tyranny,
	Aristocracy $\rightarrow$ Oligarchy, Democracy $\rightarrow$ Anarchy.
	(\d 2, \p 106)
\item 
	\keyquote{
	``These variations of government among men are due to chance. For in
	the beginningof the world, when its inhabitants were few, they lived for
	a time scattered like the beasts. Then, with the muliplication of their
	offspring, they drew together and, in order the better to be able to
	defend themselves, began to look about for a man stronger and more
	courageous than the rest, made him their head, and obeyed him'' 
	(\d 2, \p 106-107).	\comment{Dang, Hobbes!}
\item 
	``But when at a yet later stage they began to make the prince 
	hereditary instead of electing him, his heirs soon began to
	degenerate [...]. With the result thatthe prince came to be hated [...]
	which quickly brought about a tyranny.'' (\d 2, \p 107)
\item
	``civic rights'' (\p 108). No translation note.
\item
	Rome, over time,began to share power between Monarchy, Aristocracy,
	and Democracy---``the blending of these estates made a perfect
	commonwealth.'' (\d 2, \p 111)
\item
	\keyquote{
	``All writers on politics have pointed out, and throughout history there
	are plety of exammples which indicate, that in constituting and 
	legislating for a commonwealth it must needs be taken for granted that
	all men are wicked'' (\d 3, \p 112).
	}
\item
	Good order rarely appears without good fortune. (\d 4, \p 113)
\item
	\keyquote{``in every republic there are two different dispositions,
	that of the populace and that of the upper class and that all
	legislation favourable to liberty is brought about by the clash	
	between them.''} (\d 4, \p 113)
\item 
	

	
	}
\end{itemize}

\end{document}
