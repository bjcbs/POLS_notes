\documentclass[11pt]{article}

%geometry, formatting
\usepackage[margin=1.5in]{geometry}
\setlength{\parindent}{0ex}
\setlength{\parskip}{11pt}
%other packages
\usepackage{color}
\usepackage[dvipsnames]{xcolor}
\usepackage{enumitem}

%commands
\newcommand{\aside}[1]{{\color{Goldenrod}#1}}
\newcommand{\question}[1]{{\color{BurntOrange}#1}}
\newcommand{\comment}[1]{{\color{Cerulean}#1}}
\newcommand{\keyquote}[1]{{\color{BrickRed}#1}}

\newcommand{\book}[1]{{\Large {\bfseries Book #1}}}
\newcommand{\discourses}[1]{{\bfseries Discourses #1.}}
\newcommand{\discourse}[1]{{\bfseries Discourse #1.}}

\newcommand{\p}{page } 
\renewcommand{\d}{disc. }
%%%% %%%% %%%% %%%% %%%% %%%% %%%% %%%% %%%% %%%% %%%% %%%% %%%% %%%% %%%% %%%%

\begin{document}
\hfill Brent Jacobs

Notes on Machiavelli, Discourses on Livy: 
Book I (entire);
Book II, discourses 1-14, 21-23, 28-29,
33; 
Book III, discourses 1-6, 35.
Editor: Bernard Crick.

{\Large{\bfseries Dedicatory letter.}}

I had thought, in NM's dedicatory letter for \textsl{The Prince}, he was
subtley mocking Lorenzo de Medici. Now I am convinced!

\book{I}

Deals with the development of Rome and its consititution.

{\bfseries Preface.}

While the nations of antiquity, especially Rome, are
greatly admired, few try to imitate it---or even believe they can. 
This is especially true in citizenship and statemenship. So, NM will
``write a commentary on all those book of Titus Livy'' to convice men
that similar greatness can be achieved today (a relative today). (\p 98-99) 

\discourses{1-10}

\begin{itemize}
\item
	``all cities are built either by natives of the place in which they are 
	built, or by people from elsewhere''(\p 100).
\item
	Free cities should be founded in fertile places, with many resources.
	But, since this might make men grow idle, strict laws must be enacted
	to ensure the men work diligently (\p 102-104). 
\item
	``I [...] shall speak only of those [cities] which have from the outset
	been far removed from any kind of external servitude, but, instead,
	have from the start been governed in accordance with their wishes, either
	as republics or principalities'' (\d 2, \p 104-105). 

	NM does not see being a principalities as incompatible with a city
	being ``governed in accordance with their wishes.''
\item
	States must be founded with good laws. Otherwise, it is unlikely a 
	good set of laws can ever later be instituted. (\d 2, \p 105)
\item 
	Common viewpoint: Principality $\rightarrow$ Tyranny,
	Aristocracy $\rightarrow$ Oligarchy, Democracy $\rightarrow$ Anarchy.
	(\d 2, \p 106)
\item 
	\keyquote{
	``These variations of government among men are due to chance. For in
	the beginningof the world, when its inhabitants were few, they lived for
	a time scattered like the beasts. Then, with the muliplication of their
	offspring, they drew together and, in order the better to be able to
	defend themselves, began to look about for a man stronger and more
	courageous than the rest, made him their head, and obeyed him'' 
	}
    (\d 2, \p 106-107).	\comment{Dang, Hobbes!}
\item 
	``But when at a yet later stage they began to make the prince 
	hereditary instead of electing him, his heirs soon began to
	degenerate [...]. With the result thatthe prince came to be hated [...]
	which quickly brought about a tyranny.'' (\d 2, \p 107)
\item
	``civic rights'' (\p 108). No translation note.
\item
	Rome, over time,began to share power between Monarchy, Aristocracy,
	and Democracy---``the blending of these estates made a perfect
	commonwealth.'' (\d 2, \p 111)
\item
	\keyquote{
	``All writers on politics have pointed out, and throughout history there
	are plety of exammples which indicate, that in constituting and 
	legislating for a commonwealth it must needs be taken for granted that
	all men are wicked'' (\d 3, \p 112).
	}
\item
	Good order rarely appears without good fortune. (\d 4, \p 113)
\item
	\keyquote{``in every republic there are two different dispositions,
	that of the populace and that of the upper class and that all
	legislation favourable to liberty is brought about by the clash	
	between them.''} (\d 4, \p 113)
\item   
    ``And though [...] the populace may be ignorant, it is capable of 
    grasping the truth and readily yields when a man, worthy of confidence,
    lays the truth before it.'' (\d 4, \p 115). 
    \comment{Uhh... If you say so...}
\item
    Arguments could be made for entrusting either the commoners or the 
    nobility with preserving liberty. The former value it more, but 
    might cause ``endless quabbles and troubles in a republic'' (\p 116).
    (But this contradicts the previous celebration of those little 
    troubles, no?) But, comparing republics such as Venice and Rome, the
    evidence is towards entrusting the nobility with this duty. 
    (\d 5, \p 116-117)
\item
    ``for to find a middle way between the two extremes I do not think
    possible.'' (\p 123)
\item
    Republics can be set up like Venice or Sparta, preserving liberty
    is entrusted to a ruling class. This model will last longer, so
    long as the republic not try to expand. Else, the Roman model
    must be followed. (\p 120-123). 
\item 
    \question{In what sense is NM speaking of liberty? It sounds like
    both...} ``...prejudicial to the freedom of the state'' (\p 123). 
    ``
\item
    Public indictments (trials) as a way to relieve public anger 
    at individuals (\p 124-125). Justice? Ha!
\item   
    When citizens are punished in a republic, little disorder is
    caused, and civil liberties are not threatened (they are, however,
    when the punishment is made by or for foreign or private 
    interests) (\d 7, \p 125).
    \comment{Interesting notion of liberty/punishment.}     	
\item
    The citizens must be able to destroy the ambitions of a 
    too-powerful person, less disorder ensue (\d 7, \p 126).
\item
    Indictments must be supported by witnesses and evidence; else,
    they are calumnies (slander), which are detrimental to 
    the republic (\d 8, \p 129).
\item 
    In order to organize a state, one must first be the sole
    authority, then bequeth the authority to many, rather 
    than one (Sparta, Rome) (\d 9).
    \comment{This, however, contradicts
    his explanation of how Venice was founded.}   	
\item
    Tyrants are bad. (\d 10)    
\end{itemize}

\discourses{11-15}
\begin{itemize}
\item   
    ``finding the people ferocious and desiring to reduce them to
    civic obedience by means of the arts of peace, turned to
    religion as the instrument necessary'' (\d 11, \p 139).
\item
    ``the religion introduced by Numa was among the primary 
    causes of Rome's success'' (\d 11, \p 141).
\item
    \keyquote{
    ``The security of a republic or of a kingdo, therefore, deos 
    not depend upon its ruler governing it prudently during
    his lifetime, but ipon his so ordering it that, after his
    death, it may maintain itself in being.''} (\d 11, \p 142)
\end{itemize}

\discourses{16-18}
\begin{itemize}
\item
    Good institutions are foundational. Good laws cannot
    simply be laid down upon corrupt institutions. (\d 18)
\end{itemize}

\discourses{19-24}
\begin{itemize}
\item
    Look at the titles for 19 ... 23
\item 
    A well-ordered republic ``punishes [a person who has done 
    wrong] regardless of any of the good deeds he has done.''
    Good deeds must be rewarded, and bad punished. No one can
    feel their good deeds give them impunity, and only 
    punishment is a poor course. 
    (\d 24, \p 173-174)
\end{itemize}

\discourses{25-27}
\begin{itemize}
\item   
    ``He who desires or proposed to change the form of government
    in a state and wishes it to be acceptable [...] must needs retain
    at least the shadow of it's ancient custome, so
    that institutions may not appear to its people to have changes,
    though in point of fact the new institution may be radically 
    different from the old ones. This he must do because men in 
    general are as much affected by what a thing appears to be 
    as by what it is'' (\d 25, \p 174)
\item
    ``for the sort of man who is unwilling to take up [the] course of
    well doing, it is expedient [...] to enter on the path of wrong
    doing. Actually, however, most men prefer to steer a middle couse, 
    which is very harmful; for they know not how to be wholly good nor
    yet wholly bad'' (\d 26, \p 177)    
\end{itemize}

\discourses{28-32}
\begin{itemize}
\item
    Princes should avoid being ingrateful to their citizens (\d 30).
\item
    Roman generals were not punished cruely for failure, so that
    a fear of punishment would not cloud their minds and distract
    them---an important example (\d 31).
\item
    Leaders should not put off treating the people well until
    in a time of crises, when their support is needed (\d 32).
\end{itemize}

\discourses{33-36}
\begin{itemize}
\item
    ``For, if in a republic there appears some youth of noble birth
    and outstanding virtue, the eyes of every citizen at once turn towards
    him, and without further consideration they agree to show him honour''
    (\p 191). lol. 
\item 
    Successful republics need a mechanism by which, in times of crisis,
    a strong executive can take control and make swift, unquestioned 
    decisions. (\d 34)    
\end{itemize}

\discourses{37-39}
\begin{itemize}
\item
    ``If the present be compared with the remote past, it is easily
    seen that in all cities and in all peoples there are the same 
    desires and the same passions as there always were. So that,
    if one examines with diligence the past, it is easy to foresee
    the future of any commonwealth'' (\d 39, \p 207-208) 
\end{itemize}

\discourses{50-55}
\begin{itemize}
\item
    Republics, princes, etc. should, for whatever they have little
    choice but to do, make it seem like a favor (\d 51).
\item
    ``the populace, misled by the false appearance of good, often
    seeks its own ruin'' (\d 53, \p 238)
\item 
    ``gentry'', who derive wealth from land holdings they do not
    personally cultivate, ``are a pest in any republic and in any
    province'' (\d 55, \p 245-246).
\end{itemize}

\discourses{56-60}
\begin{itemize}
\item
    ``How it comes about I know not, but it is clear both from
    ancient and modern cases that no serious misfortune ever
    befalls a city or a province that has not been
    predicted either by divination or revelation or by prodigies 
    or by other heavenly signs'' (\d 56, \p 249).
    \comment{So... on that ``ancient'' versus ``modern''
    question...}
\item 
    \ldots
\end{itemize}

\book{II}
% 1-14, 21-23, 28-29, 32 33
\begin{itemize}
\item
    Only independent republics are successful! There were so
    many in the past; why not now? Because our religion has
    made us weak! The ancient religions spilled great amounts
    of blood in their rituals! Those who interpret the modern
    religion forget the glory in defending the fatherland! 
    (\d 2)
\item
    ``Hence men who in this life normally either suffer great
    adversity or enjoy great prosperity, deserve neither praise
    not blame; for one usually finds that they have been driven 
    either to ruin or to greatness by the prospect of some great
    advantage which the heavens have held out, whereby they have
    been given the chance, or have been deprived of the chance, of
    being able to act virtuously.'' (\d 29, \p 371)
\item
    NM believes that, in dealing with other states, it is important
    to avoid offense. One might violate norms! Intersting IR
    implications...
\item
    A lot about how Rome expanded. Intersting historically, but ehh
    I felt less ``political theory relevance'' 
\end{itemize}

\book{III} %1-6, 35

Look through the section titles, prepared by the editor. For Book II,
they give a quick summary of the less pithy parts.

\end{document}
