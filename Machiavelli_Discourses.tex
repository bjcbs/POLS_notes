\documentclass[11pt]{article}

%geometry, formatting
\usepackage[margin=1.5in]{geometry}
\setlength{\parindent}{0ex}
\setlength{\parskip}{11pt}
%other packages
\usepackage{color}
\usepackage[dvipsnames]{xcolor}
\usepackage{enumitem}

%commands
\newcommand{\aside}[1]{{\color{Goldenrod}#1}}
\newcommand{\question}[1]{{\color{BurntOrange}#1}}
\newcommand{\comment}[1]{{\color{Cerulean}#1}}
\newcommand{\keyquote}[1]{{\color{BrickRed}#1}}

\newcommand{\book}[1]{{\Large {\bfseries Book #1}}}
\newcommand{\discourse}[1]{{\bfseries Discourse #1.}}

\newcommand{\p}{page } 
%%%% %%%% %%%% %%%% %%%% %%%% %%%% %%%% %%%% %%%% %%%% %%%% %%%% %%%% %%%% %%%%

\begin{document}
\hfill Brent Jacobs

Notes on Machiavelli, Discourses on Livy: 
Book I (entire);
Book II, discourses 1-14, 21-23, 28-29,
33; 
Book III, discourses 1-6, 35.
Editor: Bernard Crick.

{\Large{\bfseries Dedicatory letter.}}

I had thought, in NM's dedicatory letter for \textsl{The Prince}, he was
subtley mocking Lorenzo de Medici. Now I am convinced!

\book{I}

Deals with the development of Rome and its consititution.

{\bfseries Preface.}

While the nations of antiquity, especially Rome, are
greatly admired, few try to imitate it---or even believe they can. 
This is especially true in citizenship and statemenship. So, NM will
``write a commentary on all those book of Titus Livy'' to convice men
that similar greatness can be achieved today (a relative today). (\p 98-99) 

\discourse{1}

\begin{itemize}
\item
	``all cities are built either by natives of the place in which they are 
	built, or by people from elsewhere''(\p 100). 
\end{itemize}

\end{document}
